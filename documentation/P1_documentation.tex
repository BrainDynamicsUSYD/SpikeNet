\documentclass{article}
\usepackage{amsmath}
\usepackage{cite}
\usepackage{tabulary}
\usepackage{listings}

\title{Project1 Documentation}
\author{Yifan Gu  \\
	Complex System Group, School of Physics, USYD  \\
	}

\date{\today} 

\begin{document}
 \maketitle


\section{Model}

\numberwithin{equation}{section}

Neurons are modeled as conductance-based leaky integrate-and-fire units. 
The membrane potential $V^{\alpha}_{i}$ of the $i$-th neuron ($i=1,\cdots,N_{\alpha}$) from population $\alpha=E,I,X$ (excitatory, inhibitory and external population, respectively) evolves according to

\begin{equation}
C_{m}\frac{dV^{\alpha}_{i}}{dt} = -g_{L}(V^{\alpha}_{i}-V_{L}) + I^{\alpha}_{i,syn}(t) + I^{\alpha}_{i,app}(t), \text{ if } V^{\alpha}_{i} < V_{th}
\end{equation}

When neurons reach the threshold $V_{th}$, a spike is emitted and they are reset to $V_{rt}$ for an absolute refractory period $\tau_{ref}$. 
The spike times $t^{\alpha}_{i}$ are recorded.

The synaptic currents received by a neuron are given by


\begin{align}
I^{\alpha}_{i,syn}(t) &= \sum_{\beta=1}^{P}I^{\alpha\beta}_{i}(t) \\
 &= \sum_{\beta=1}^{P}[-g_{i}^{\alpha\beta}(V^{\alpha}_{i}-V_{rev}^{\beta})] \\
 &= \sum_{\beta=1}^{P}\{-[\sum_{j=1}^{N_{\beta}}a_{ij}^{\alpha\beta}g_{ij}^{\alpha\beta}s_{ij}^{\alpha\beta}(t)](V^{\alpha}_{i}-V_{rev}^{\beta})\}\label{eq:I_syn}
\end{align}


\noindent
where $V_{rev}^{\beta}$ is the reversal potential of the corresponding current $I^{\alpha\beta}_{i}$ induced by pre-synaptic population $\beta$ and mathematically models the excitatory/inhibitory nature of it. 
$a_{ij}^{\alpha\beta}$ is a binary variable which determines the existence of synapse from the $j$-th neuron in population $\beta$ to the $i$-th neuron in  population $\alpha$, while $g_{ij}^{\alpha\beta}$ reflects the (maximal) strength of the synaptic conductance. 
The gating variable $s_{ij}^{\alpha\beta}(t)$ models the instananeous value of synaptic conductance in terms of the fraction of open channels, described by

\begin{equation}
\frac{ds_{ij}^{\alpha\beta}}{dt} = -\frac{s_{ij}^{\alpha\beta}}{\tau_{d}^{\beta}}+\sum_{t^{\beta}_{j}}h^{\beta}(t-t^{\beta}_{j}-d^{\alpha\beta}_{ij})(1-s_{ij}^{\alpha\beta})
\label{eq:gating}
\end{equation}

\noindent
where $h$ models the concentration time-course of the channel-opening neurotransmitters, arrived with a conduction delay $d^{\alpha\beta}_{ij}$ after the pre-synaptic spike time $t^{\beta}_{j}$. 
The $(1-s_{ij}^{\alpha\beta})$ term introduces saturation effect. 
Following the simplification in \cite{destexhe1994efficient}, a rectangular pulse with unitary area is used for $h$

\begin{equation}
h(t) = 
\begin{cases}
1/\tau_{r},& \text{ if } 0\leq t \leq \tau_{r}\\
0,& \text{ otherwise}
\end{cases}
\end{equation}


The default parameters are summarized in Table \ref{table:model constants}. 
All numerical values are in consistant units unless mentioned otherwise (ms for time, mV for voltage, nA for current, nF for capacitance and ${\mu}$S for conductance). 
Numerical integration is performed using Euler method with a time-step of 0.1 ms\cite{litwin2012slow}.



\begin{table}
\begin{center}
  \begin{tabulary}{3.0\textwidth}{RRR}
	$\alpha$  & E & I \\
	\hline
	$\tau_{r}^{\alpha}$ (ms) & 1 & 1\\
	$\tau_{d}^{\alpha}$ (ms) & 5 & 3 \\
	$V_{rev}^{\alpha}$ (mV) & 0 & -80 \\
	
	$\tau_{ref}$ (ms) & \multicolumn{2}{c}{2} \\
	$C_{m}$ (nF) & \multicolumn{2}{c}{0.25} \\
	$g_{L}$ ($\mu$S) & \multicolumn{2}{c}{0.0167} \\
	$V_{L}$ (mV) & \multicolumn{2}{c}{-70} \\
	$V_{rt}$ (mV) & \multicolumn{2}{c}{-60} \\
	$\theta$ (mV) & \multicolumn{2}{c}{-50} \\
  \end{tabulary}  
  \caption{Model constants}
  \label{table:model constants}
\end{center}
\end{table}




\section{Input File Format}

The input files are text files.
The default input filename is ``input\_filename.ygin''.
It is advised that synapse definitions always be given in a separate file.
The default synapse definition filename is ``input\_filename.ygin\_syn''.
Non-default synapse definition file path and name can be specified by ``SYNF001'' command.
Input data format is as following.

\lstset{
  basicstyle=\fontsize{9}{11}\selectfont\ttfamily
}



\begin{lstlisting}[mathescape]

> INIT001 # number of neurons in each population
	N1, N2, ...,

> INIT002 # time step length and total number of steps
	dt (ms), step_tot,

> INIT003 # random initial distributions for V
	p_fire_pop1 (range (0,1]), ..., p_fire_popN, 

> INIT004 # external Gaussian currents
	pop_ind, mean (nA), std (nA), 

> INIT005 # external Poissonian spikes
	pop_ind, type_ext, K_ext ($\mu$S), Num_ext, ia, ib,
	rate_1 (Hz), rate_2, ..., rate_step_tot

> INIT006 # chemical connection definition
	type, pop_ind_pre, pop_ind_post,
	I (row vector),
	J (row vector),
	K (row vector),
	D (row vector),

> SYNF001 # non-default synapse definition file name
	path/to/file_name (no comma!)

> KILL001 # runaway killer setting
	pop_ind, min_ms, runaway_Hz, Hz_ms,

> PARA001 # non-default neuron population parameter
	pop_ind, number_of_parameters,
	parameter_name1, value1,
	parameter_name2, value2,
	...

> PARA002 # non-default synapse parameter
	number_of_parameters,
	parameter_name1, value1,
	parameter_name2, value2,
	...

> SAMP001 # neuronal data sampling 
	pop_ind, sample_step_ms,
	data_type (logical vector),
	ind1, ind2, ..., indX, (sample neuron indices)
	# Note that data_type specifies sample data types
	# and it must correspond to 
	# [V,I_leak,I_AMPA,I_GABA,I_NMDA,I_GJ,I_ext]

> SAMP002 # populational data sampling
	pop_ind,
	data_type (logical vector),
	1, 1, 0, 0, ..., (1-by-step_tot logical values)
	# Note that data_type specifies sample data types
	# and it must correspond to 
	# [V,I_leak,I_AMPA,I_GABA,I_NMDA,I_GJ,I_ext]

\end{lstlisting}

\section{Output File Format}
The output files are text files.
The default output filename is ``input\_filename-time\_stamp.ygout''.
For data completeness, the corresponding input file will be attached to the output file.
The output data format is as following.
\begin{lstlisting}

> KILL002 # step at which runaway activity is killed
	step_killed,

> POPD001 # spike history of neuron population
	pop_ind,
	spike_neuron_ind (row vector),
	num_spikes_t (1-by-step_tot row vector),
	num_ref_t (1-by-step_tot row vector),

> POPD002 # neuron parameters in the population
	pop_ind, number_of_parameters,
	parameter_name1, value1,
	parameter_name2, value2,
	...

> POPD003 # sampled populational data
	pop_ind, number_of_time_points,
	data_name1, ..., data_nameX,
	data_1 (all_neurons-by-sampled_steps matrix),
	data_2 (all_neurons-by-sampled_steps matrix),
	...
	data_X (all_neurons-by-sampled_steps matrix),

> POPD004 # sampled neuronal data
	pop_ind, number_of_sample_neurons,
	data_name1, ..., data_nameX,
	data_1 (sampled_neurons-by-sampled_steps matrix),
	data_2 (sampled_neurons-by-sampled_steps matrix),
	...
	data_X (sampled_neurons-by-sampled_steps matrix),

> SYND001 # synaptic connection parameters
	number_of_parameters,
	parameter_name1, value1,
	parameter_name2, value2,
	...


\end{lstlisting}

\bibliography{P1_documentation}{}
\bibliographystyle{plain}



\end{document}
